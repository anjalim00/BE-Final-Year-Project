\chapter{Project Concept}
\section{Objectives}
\begin{itemize}
    \item To generate a summary of a text/paragraph by providing the input in the form of a paragraph or a file.
    \item To generate bullet/key points of a paragraph that would cover only the important points and won’t be as long as a summary.
    \item Summarization of a text/paragraph would be achieved by extractive and abstractive approaches.
\end{itemize}

  


\section{Literature Review}
\textbf{\textit{   Title : Text summarization using neural networks
Authors : Mr Anish Jadhav, Mr Rajat Jain, Mr Steve Fernandes, Mrs Sana Shaikh
Year of publication : 2019}}\\
Extractive Text Summarization is the method of extracting content from the document and combining it to form a text smaller in size. This ensures that only the words having relevance in the document are selected for the summarization. 
Whereas, Abstractive Text Summarization is capable of depicting information by creating new sentences. It can be divided into Structured and Semantic approaches, each of which  can be subdivided into subcategories based on various methods.
The methods are:\\
•Tree-based approach\\
•Ontology-based approach\\
•Rule-based approach\\
•Graph-based approach  \\      

\textbf{\textit{Title : Extractive text summarization using sentence ranking
Authors : Mrs J.N.Madhuri, Mr Ganesh Kumar
Year of publication : 2019}}\\
Automatic Text summarization is the technique to identify the most useful and necessary information in a text. It has two approaches 1) Abstractive text summarization and 2) Extractive text summarization. An extractive text summarization means an important information or sentence are extracted from the given text file or original document. In this paper, a novel statistical method to perform an extractive text summarization on single document is demonstrated. The method extraction of sentences, which gives the idea of the input text in a short form, is presented.\\
 
\textbf{\textit{Title : An overview on extractive text summarization
Authors : Mr Shohreh Rad Ramini, Mr Ali Toofanzahdeh Mozhdehi
Year of publications : 2017}}\\
Text summarization is the process of automatically creating and condensing form of a given document and preserving its information content source into a shorter version with overall meaning. According to difference requirements summary with respect to input text, established summarization systems should be created and classified based on the type of input text. In this study, at first, the topic of text mining and its relationship with text summarization are considered.Then a review has been done on some of the summarization approaches and their important parameters for extracting predominant sentences.
\\

\section{Problem Definition}
The need for text summarization is continuously increasing as today's world is getting flooded with a growing number of articles and links to choose from with the expansion of the internet. 
Human beings tend to read the whole document to develop an understanding of it and generate a summary by keeping the main points in mind. It is getting extremely difficult to obtain the required information from this pool of words and sentences in a short period. 
Going through all the documents, articles, and different forms of information to manually summarize is extremely time-consuming and exhausting for humans. Summarization helps in saving valuable time and conveys the main essence from which the reader can decide if they want to dig deeper.

\section{Scope}
The aim of this project is to achieve automation of generating a summary for the given set of data by generating a summarized text of fixed word length by extractive summarization techniques. 
The model designed in the project will be trained such that it will choose important words and sentences from the input text and arrange them to formulate meaningful sentences. 
We will be implementing abstractive summarization as well where the ambiguity of sentences in the summary will be reduced as this approach generates a summary by framing new sentences that serve the purpose.  
The existing summarization tools have a restriction on the word length for input text so we will be working on this aspect, and try to remove such barriers.

\section{Technology Stack}
\begin{itemize}
\item Python 3.8
\item Pandas
\item Numpy
\item NLP
\item NLTK
\item RNN
\item LSTM
\item MySQL
\item HTML, Php

\end{itemize}

\section{Benefits for Society}

\begin{itemize}
\item Summaries reduce reading time.
\item When researching documents, summaries make the selection process easier.
\item Automatic summarization improves the effectiveness of indexing.
\item Personalized summaries are useful in question-answering systems as they provide personalized information.
\item Using automatic or semi-automatic summarization systems enables commercial abstract services to increase the number of texts they are able to process.

\end{itemize}