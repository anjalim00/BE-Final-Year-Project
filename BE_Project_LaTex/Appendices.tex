\chapter*{Appendices}
\addcontentsline{toc}{chapter}{Appendices}
Detailed information, lengthy derivations, raw experimental observations etc. are to be
presented in the separate appendices, which shall be numbered in Roman Capitals (e.g.
“Appendix I”). Since reference can be drawn to published/unpublished literature in the
appendices these should precede the “Literature Cited” section.\\

%section*{Appendices}
\section*{Appendix-A: NS2 Download and Installation}
\addcontentsline{toc}{section}{Appendix-A}
1. Download ns-allinone-2.35.tar.gz from http://sourceforge.net/projects/nsnam/\\
\\
2. Place ns-allinone-2.35.tar in your desired directory; like /home/vishal.\\
\\
3. Go to terminal and do as following commands\\
\textbf{sudo apt-get update}\\
\textbf{sudo apt-get install automake autoconf libxmu-dev build-essential}\\
\\
4. Extract ns-allinone-2.35 and after extracting go to folder ns-allinone-2.35 from Terminal as\\
\textbf{\$cd ns-allinone-2.35}\\
\textbf{\$./install}\\
\\
5. Path Setting\\
\textbf{\$ gedit .bashrc}\\
\\
This command will open an existing file in editor. Just put the following path which is given bellow. [Remember that our ns-allinone path is /home/vishal. we will change this path according to our ns-allinone folder's path]\\
\\
export PATH=\$PATH:/home/vishal/ns-allinone-2.35/bin:/home/vishal/ns-allinone-2.35/tcl8.5.10/\\
unix/home/vishal/ns-allinone-2.35/tk8.5.10/unix\\
\\
export LD\_LIBRARY\_PATH=\$LD\_LIBRARY\_PATH:/home/vishal/ns-allinone- 2.35/otcl-1.14:/home/\\
vishal/ns-allinone-2.35/lib\\
\\
export TCL\_LIRARY\_PATH=\$TCL\_LIBRARY\_PATH:/home/vishal/ns-allinone-2.35/tcl8.5.10/library\\
\\
After this save and exit.\\
\\
6. Now type in terminal to check that, is all command we entered in .bashrc is correct or not? And To take the effect immediately\\
\textbf{\$source .bashrc}\\
\\
7. Then perform the validation test using this command.\\
\textbf{\$ ./validate}\\
\\
8. Run ns2 using this command\\
\textbf{\$ns}\\
\\
We will get \% prompt in our terminal. Now ns2 has been installed.\\

